\documentclass {article} 

\usepackage{lmodern}
\usepackage [spanish] {babel} 
\usepackage [T1]{fontenc}
\usepackage [latin1]{inputenc}
\usepackage{amsthm} % para poder usar newtheorem
\usepackage{cancel} %Para poder hacer el simbolo "no es consecuencia sem�ntica" etc.
\usepackage{graphicx} 
\usepackage{amsmath} %para poder usar mathbb
\usepackage{amsfonts} %sigo intentando usar mathbb
\usepackage{amssymb} %therefore
\usepackage{ mathabx } %comillas
\usepackage{ verbatim } 
\addto\captionsspanish{%
\def\bibname{Referencias}%
\def\tablename{Cuadro}%
}

\theoremstyle{remark}
\newtheorem{thm}{Teorema}
\newtheorem{lem}{Lema}[section]
\newtheorem{cor}{Corolario}[section]
\newtheorem{deff}{Definici�n}[section]
\newtheorem{obs}{Observaci�n}[section]
\newtheorem{ej}{Ejercicio}[section]
\newtheorem{ex}{Ejemplo}[section]
\newtheorem{alg}{Algoritmo}[section]
\usepackage[latin1]{inputenc} 
\usepackage{listings}
\lstset{language=Haskell, breaklines=true, basicstyle=\footnotesize} %paquete para poner c�digo
\usepackage{proof}


\usepackage{color}

\definecolor{mygreen}{rgb}{0,0.6,0}
\definecolor{mygray}{rgb}{0.5,0.5,0.5}
\definecolor{mymauve}{rgb}{0.58,0,0.82}

\lstset{ %
  backgroundcolor=\color{white},   % choose the background color; you must add \usepackage{color} or \usepackage{xcolor}
  basicstyle=\footnotesize,        % the size of the fonts that are used for the code
  breakatwhitespace=false,         % sets if automatic breaks should only happen at whitespace
  breaklines=true,                 % sets automatic line breaking
  captionpos=b,                    % sets the caption-position to bottom
  commentstyle=\color{mygreen},    % comment style
  deletekeywords={...},            % if you want to delete keywords from the given language
  escapeinside={\%*}{*)},          % if you want to add LaTeX within your code
  extendedchars=true,              % lets you use non-ASCII characters; for 8-bits encodings only, does not work with UTF-8
  frame=single,                    % adds a frame around the code
  keepspaces=true,                 % keeps spaces in text, useful for keeping indentation of code (possibly needs columns=flexible)
  keywordstyle=\color{blue},       % keyword style
  language=Haskell,                 % the language of the code
  otherkeywords={*,...},            % if you want to add more keywords to the set
  numbers=left,                    % where to put the line-numbers; possible values are (none, left, right)
  numbersep=5pt,                   % how far the line-numbers are from the code
  numberstyle=\color{black}, % the style that is used for the line-numbers
  rulecolor=\color{black},         % if not set, the frame-color may be changed on line-breaks within not-black text (e.g. comments (green here))
  showspaces=false,                % show spaces everywhere adding particular underscores; it overrides 'showstringspaces'
  showstringspaces=false,          % underline spaces within strings only
  showtabs=false,                  % show tabs within strings adding particular underscores
  stepnumber=1,                    % the step between two line-numbers. If it's 1, each line will be numbered
  stringstyle=\color{mymauve},     % string literal style
  tabsize=2,                       % sets default tabsize to 2 spaces
  title=\lstname                   % show the filename of files included with \lstinputlisting; also try caption instead of title
}



\usepackage{anysize}
\marginsize{3cm}{3cm}{3cm}{3cm}










\begin{document} 

\title{TRABAJO PR�CTICO IV }
\author{
Meli Sebasti�n. Rodr�guez Jerem�as. \\ \\
\small{\texttt{AN�LISIS DE LENGUAJES DE PROGRAMACI�N}}
}
\date{\small{\today}}

\maketitle

\newpage 
\section*{Ejercicio 1}
\subsection*{Ejercicio 1.a}

\par Dado el constructor de tipos State y la siguiente instancia a la clase m�nada:
\begin{verbatim}
newtype State a = State { runState :: Env -> (a, Env) }

instance Monad State where
    return x = State (\s -> (x, s))
    m >>= f = State (\s -> let (v, s') = runState m s in
                           runState (f v) s')

\end{verbatim}
\par Debemos demostrar que es una instancia v�lida.

\begin{itemize}
	\item \textbf{Monad 1} 
	\begin{verbatim}
    return x >>= f
=<def de return>
    State (\e->(x,e)) >>= f
=<def de >>=>
    State (\s -> let (x',s') = runState (State (\e->(x,e))) s 
              in  runState (f x') s')
=<def de runstate>						
    State (\s -> let (x',s') = (\e->(x,e)) s 
              in  runState (f x') s')
=< aplicaci�n de funci�n >
    State (\s -> let (x',s') = (x,s) 
              in  runState (f x') s')
=< aplicaci�n de let >
    State (\s ->  runState (f x) s )
=<*>
    State (\s ->  runState (State p) s )
=<def de runStateate>
    State (\s -> p s)
=<extensionalidad>
    State p
= < * >
    f x
    
	(*) f x = State p 
\end{verbatim}
	\item \textbf{Monad 2} 
	\begin{verbatim}
	
    (State h) >>= return 
= < def de >>= y de runState >
    State (\s -> let (x,s') = h s 
               in   runState (return x) s' )
= < def de return >
    State (\s -> let (x,s') = h s 
               in   runState (State (\e -> (x,e))) s' )
= <def de runStateate y aplicaci�n >
    State (\s -> let (x,s') = h s 
               in  (x,s') ) 
= <extensionalidad>
    State  (\s -> h s)
= < extensionalidad>
    State h
    
	\end {verbatim}
	
	\item \textbf{Monad 3} 
	
	\begin{verbatim}
	
    (State h >>= f) >>= g
=< >>= y def de runState>
    (State (\s -> let (x,s') = h s in   runState (f x) s' ) ) >>= g
= <*>
    (State (\s -> let (x,s') = (a0,s0) in   runState (f x) s' ) ) >>= g
=<aplico let>
    (State (\s -> runState (f a0) s0 ) ) >>= g
= <*>
    (State (\s -> runState (State kf) s0 ) ) >>= g
= < def de runState >
    (State (\s -> kf s0) ) >>= g
= < >>= y def de runState>
    State (\e -> let (z0,z1) = (\s -> kf s0) e in runState (g z0) z1)
= < aplico >
    State (\e -> let (z0,z1) = kf s0 in runState (g z0) z1)
= <*>
    State (\e -> let (z0,z1) = (a1,s1) in runState (g z0) z1)
= < evaluo let>
    State (\e -> runState (g a1) s1)       
= < evaluacion de let >     
    State (\s ->    let (p0,p1) = (a1,s1) in runState (g p0) p1    )       
= < * >     
    State (\s ->    let (p0,p1) = kf s0 in runState (g p0) p1    ) 
= < runState def >
    State (\s -> runState ( State (\e -> let (p0,p1) = kf e in runState (g p0) p1 )  ) s0)
=< >>= >
    State (\s -> runState ((State kf) >>=g) s0)
= <*>
    State (\s -> runState (f a0 >>=g) s0)
=<aplico let>
    State (\s -> let (x,s') = (a0,s0) in runState (f x >>=g) s')
=<*>
    State (\s -> let (x,s') = h s in runState (f x >>=g) s')
= < app> 
    State (\s -> let (x,s') = h s in runState ( (\x'-> f x' >>=g) x) s' )
=< >>= y def de runState>
    State h >>= (\x -> f x >>=g)


*) h s = (a0,s0)
*) f a0 = (State kf) 
*) kf s0 = (a1,s1)             
	
	
\end {verbatim}
	
\end{itemize}

\subsection*{Ejercicio 1.b}
\par Ver en Eval1.hs

\newpage
\section* {Ejercicio 2}
\subsection*{Ejercicio 2.a}
\begin{verbatim}
newtype StateError a = StateError { runStateError :: Env -> Maybe (a, Env) }

instance Monad StateError where
        return x           = StateError (\e -> Just (x,e))
        StateError h >>= f = StateError (\e -> case (h e) of
                                                  Nothing     -> Nothing
                                                  Just (a,e') -> runStateError (f a) e' )
\end{verbatim}

\subsection*{Ejercicio 2.b}
\begin{verbatim}
instance MonadError StateError where
        throw = StateError (\e -> Nothing)
\end{verbatim}

\subsection*{Ejercicio 2.c}
\begin{verbatim}
instance MonadState StateError where
    lookfor v = StateError (\s -> Just (lookfor' v s, s))
                where lookfor' v ((u, j):ss) | v == u = j
                                             | v /= u = lookfor' v ss
        -- Suponemos que no se utilizan variables no declaradas en LIS
    update v i = StateError (\s -> Just ((), update' v i s))
                 where update' v i [] = [(v, i)]
                       update' v i ((u, _):ss) | v == u = (v, i):ss
                       update' v i ((u, j):ss) | v /= u = (u, j):(update' v i ss)

\end{verbatim}
\subsection*{Ejercicio 2.d}

\par Ver en Eval2.hs
\newpage
\section* {Ejercicio 3}
\subsection*{Ejercicio 3.a}
\begin{verbatim}
newtype StateErrorTick a = StateErrorTick { runStateErrorTick :: Env -> (Maybe (a, Env),Int) }
--Agregamos un Int para contar la cantidad de operaciones. Si se produce un error 
--contamos las operaciones hasta el error.
\end{verbatim}
\subsection*{Ejercicio 3.b}
\begin{verbatim}
class Monad m => MonadTick m where
        tick :: m ()
\end{verbatim}
\subsection*{Ejercicio 3.c}
\begin{verbatim}
instance MonadTick StateErrorTick where
        tick = StateErrorTick (\e -> (Just ( () , e) , 1))
\end{verbatim}
\subsection*{Ejercicio 3.d}
\begin{verbatim}
instance MonadError StateErrorTick where
        throw = StateErrorTick (\e -> (Nothing,0))
\end{verbatim}
\subsection*{Ejercicio 3.e}
\begin{verbatim}
instance MonadState StateErrorTick where
    lookfor v = StateErrorTick (\s -> (Just (lookfor' v s, s),0) )
                where lookfor' v ((u, j):ss) | v == u = j
                                             | v /= u = lookfor' v ss
        -- Suponemos que no se utilizan variables no declaradas en LIS
    update v i = StateErrorTick (\s -> (Just ((), update' v i s),0))
                 where update' v i [] = [(v, i)]
                       update' v i ((u, _):ss) | v == u = (v, i):ss
                       update' v i ((u, j):ss) | v /= u = (u, j):(update' v i ss)
\end{verbatim}
\subsection*{Ejercicio 3.f}

\par Ver en Eval3.hs

\end{document}


