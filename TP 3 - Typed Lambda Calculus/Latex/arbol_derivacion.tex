\documentclass[12pt,a2paper]{article} 

\usepackage{lmodern}
\usepackage [spanish] {babel} 
\usepackage [T1]{fontenc}
\usepackage [latin1]{inputenc}
\usepackage{amsthm} % para poder usar newtheorem
\usepackage{cancel} %Para poder hacer el simbolo "no es consecuencia semántica" etc.
\usepackage{graphicx} 
\usepackage{amsmath} %para poder usar mathbb
\usepackage{amsfonts} %sigo intentando usar mathbb
\usepackage{amssymb} %therefore
\usepackage{ mathabx } %comillas
\usepackage[latin1]{inputenc} 
\usepackage[landscape]{geometry}
\usepackage{proof}










\begin{document} 


\section*{Ejercicio 1}



\infer[(T-ABS)]{ \vdash ( \ \lambda x:B\rightarrow B\rightarrow B.\lambda y:B\rightarrow B.\lambda z:B.(x \ z) \ (y \ z) \ ) : ( B \rightarrow B\rightarrow B) \rightarrow  (B \rightarrow  B) \rightarrow  B \rightarrow  B  }{
		\infer[(T-ABS)]{x:B\rightarrow B\rightarrow B \vdash  (\lambda y:B\rightarrow B.\lambda z:B.(x \ z) \ (y \ z) \ ) : (B \rightarrow  B) \rightarrow  B \rightarrow  B  }{
			\infer[(T-ABS)]{x:B\rightarrow B\rightarrow B,y:B\rightarrow B \vdash  (\lambda z:B.(x \ z) \ (y \ z) \ ) :  B \rightarrow  B  }{
				\infer[(T-APP)] {x:B\rightarrow B\rightarrow B,y:B\rightarrow B, z:B \vdash ( (x \ z) \ (y \ z)) : B  }{
					\infer[(T-APP)]{x:B\rightarrow B\rightarrow B,y:B\rightarrow B, z:B \vdash (x \ z) : B \rightarrow B }{ 
						\infer[(T-VAR)] {x:B\rightarrow B\rightarrow B,y:B\rightarrow B, z:B \vdash x:B\rightarrow B\rightarrow B }{} 
						&
						\infer[(T-VAR)]{x:B\rightarrow B\rightarrow B,y:B\rightarrow B, z:B \vdash z:B}{}
					} }
			&
			\infer[(T-APP)] {x:B\rightarrow B\rightarrow B,y:B\rightarrow B, z:B \vdash (y \ z) : B}{ 
			\infer[(T-VAR)]{x:B\rightarrow B\rightarrow B,y:B\rightarrow B, z:B \vdash y : B \rightarrow B }{}
			&
			\infer[(T-VAR)]{x:B\rightarrow B\rightarrow B,y:B\rightarrow B, z:B \vdash z : B  }{}
			}
			}
					} }



\section*{Ejercicio 5}


\infer[T-ASCRIBE]{ \vdash (let \ z = ( (\lambda x :  \ B . x)  \ as \  B \rightarrow B) \  in \  z)  \ as  \ B \rightarrow B :  B \rightarrow B}{
 			\infer[T-LET] {\vdash let \ z = ( (\lambda x :  \ B . x)  \ as \  B \rightarrow B) \  in \  z :  B \rightarrow B}{
				\infer[T-ASCRIBE] {\vdash(\lambda x :  \ B . x)  \ as \  B \rightarrow B : B \rightarrow B }{ 
					\infer[T-ABS]{\vdash(\lambda x :  \ B . x) :  B \rightarrow B }{
						\infer[T-VAR] {x:B \vdash x:B }{}
						}}
				&
				\infer[T-VAR]{z: B \rightarrow B \vdash z: B \rightarrow B}{} 
				}}

\section*{Ejercicio 9}

\infer[T-FST]{\vdash fst (unit \ as  \ Unit ,  \ \lambda x : (B,B)  \  . snd \  x) : Unit }{
	\infer[T-PAIR] {\vdash (unit \ as  \ Unit ,  \ \lambda x : (B,B)  \  . snd \  x) : (Unit,(B,B)\rightarrow B) }{
		\infer[T-ASCRIBE] { \vdash unit \ as \ Unit : Unit}{ \vdash \infer[T-UNIT]{unit : Unit}{}}
		&
		\infer[T-ABS] { \vdash \lambda x : (B,B)  \  . snd \  x : (B,B)  \rightarrow B }{ 
			\infer[T-SND]{ x:(B,B) \vdash snd \  x : B }{\infer[T-VAR]{x:(B,B) \vdash x:(B,B)}{}} }
	}}


\end{document}
