\documentclass {article} 

\usepackage{lmodern}
\usepackage [spanish] {babel} 
\usepackage [T1]{fontenc}
\usepackage [latin1]{inputenc}
\usepackage{amsthm} % para poder usar newtheorem
\usepackage{cancel} %Para poder hacer el simbolo "no es consecuencia sem�ntica" etc.
\usepackage{graphicx} 
\usepackage{amsmath} %para poder usar mathbb
\usepackage{amsfonts} %sigo intentando usar mathbb
\usepackage{amssymb} %therefore
\usepackage{ mathabx } %comillas

\addto\captionsspanish{%
\def\bibname{Referencias}%
\def\tablename{Cuadro}%
}

\theoremstyle{remark}
\newtheorem{thm}{Teorema}
\newtheorem{lem}{Lema}[section]
\newtheorem{cor}{Corolario}[section]
\newtheorem{deff}{Definici�n}[section]
\newtheorem{obs}{Observaci�n}[section]
\newtheorem{ej}{Ejercicio}[section]
\newtheorem{ex}{Ejemplo}[section]
\newtheorem{alg}{Algoritmo}[section]
\usepackage[latin1]{inputenc} 
\usepackage{listings}
\lstset{language=Haskell, breaklines=true, basicstyle=\footnotesize} %paquete para poner c�digo
\usepackage{proof}


\usepackage{color}

\definecolor{mygreen}{rgb}{0,0.6,0}
\definecolor{mygray}{rgb}{0.5,0.5,0.5}
\definecolor{mymauve}{rgb}{0.58,0,0.82}

\lstset{ %
  backgroundcolor=\color{white},   % choose the background color; you must add \usepackage{color} or \usepackage{xcolor}
  basicstyle=\footnotesize,        % the size of the fonts that are used for the code
  breakatwhitespace=false,         % sets if automatic breaks should only happen at whitespace
  breaklines=true,                 % sets automatic line breaking
  captionpos=b,                    % sets the caption-position to bottom
  commentstyle=\color{mygreen},    % comment style
  deletekeywords={...},            % if you want to delete keywords from the given language
  escapeinside={\%*}{*)},          % if you want to add LaTeX within your code
  extendedchars=true,              % lets you use non-ASCII characters; for 8-bits encodings only, does not work with UTF-8
  frame=single,                    % adds a frame around the code
  keepspaces=true,                 % keeps spaces in text, useful for keeping indentation of code (possibly needs columns=flexible)
  keywordstyle=\color{blue},       % keyword style
  language=Haskell,                 % the language of the code
  otherkeywords={*,...},            % if you want to add more keywords to the set
  numbers=left,                    % where to put the line-numbers; possible values are (none, left, right)
  numbersep=5pt,                   % how far the line-numbers are from the code
  numberstyle=\color{black}, % the style that is used for the line-numbers
  rulecolor=\color{black},         % if not set, the frame-color may be changed on line-breaks within not-black text (e.g. comments (green here))
  showspaces=false,                % show spaces everywhere adding particular underscores; it overrides 'showstringspaces'
  showstringspaces=false,          % underline spaces within strings only
  showtabs=false,                  % show tabs within strings adding particular underscores
  stepnumber=1,                    % the step between two line-numbers. If it's 1, each line will be numbered
  stringstyle=\color{mymauve},     % string literal style
  tabsize=2,                       % sets default tabsize to 2 spaces
  title=\lstname                   % show the filename of files included with \lstinputlisting; also try caption instead of title
}



\usepackage{anysize}
\marginsize{3cm}{3cm}{3cm}{3cm}




\begin{document}

\title{TRABAJO PR�CTICO III}
\author{
Meli Sebasti�n. Rodr�guez Jerem�as. \\ \\
\small{\texttt{An�lisis de Lenguajes de Programaci�n}}
}
\date{\small{\today}}

\maketitle


\newpage

 
\section*{Ejercicio 1}
La derivaci�n se encuentra en el archivo derivaciones.pdf .

\section*{Ejercicio 2}

\par La funci�n infer retorna un valor del tipo Either String Type porque, dado un t�rmino, se dar� uno de los siguientes dos casos: Que tenga un tipo v�lido o no. Si no puede asignarle ning�n tipo a un t�rmino, devuelve un left string detallando el error; caso contrario (el t�rmino tiene un tipo v�lido), utiliza el constructor right.

\par Si el tipo de retorno de infer fuera Type, tendr�amos que agregar un tipo Error y todas las reglas para este tipo, para devolver un resultado adecuado al inferir tipos de expresiones mal tipadas; o dise�ar otra forma de se�alar las expresiones mal tipadas y alg�n detalle del error.

\par El operador $$(>>=)$$ tiene este tipo:
	
	\begin{center}
	$(>>=)$ :: Either String Type -> (Type -> Either String Type) -> Either String Type
	\end{center}
y esta definici�n
\begin{center}
$(>>=)$ v f = either Left f v
\end{center}

Donde:

\begin{flushleft}

either (Left  x) f g = f x\\
either (Right x) f g = g x\\
\end{flushleft}

Entonces $(>>=)$ toma un valor del tipo Either String Type y analiza su estructura:
\begin{itemize}
	\item Si es de la forma Left  xs, retorna Left xs (lo mismo que recibi�).
  \item Si es de la forma Right r, retorna f r , donde f es una funcion que recive un Type y devuelve un Either String Type.
\end{itemize}

Esto nos sirve cuando queremos inferir el tipo de algun t�rmino, y precisamos inferir recursivamente el tipo de subt�rminos. En este caso podemos utilizar $(>>=)$ para manejar las situaciones en que subt�rminos est�n mal tipados (o no).


\section*{Ejercicio 3}
Para a�adir la construcci�n let, utilizamos un nuevo constructor de datos en el tipo LamTerm:
\begin{center}
Let String LamTerm LamTerm
\end{center}
Modificamos el parser para capturar las expresiones let en este constructutor de LamTerm. Luego, decidimos modificar s�lamente la funci�n de conversi�n para interpretar un LamTer let x = e in t  como un Term que represente aplici�n ($\lambda$ x : T . t) e; utilizando infer para obtener T.
El resto de las funciones no precisan ser modificadas.

\section*{Ejercicio 4}
Modificamos todas las funciones para extender al operador as.

\section*{Ejercicio 5}
Ver en derivaciones.pdf

\section*{Ejercicio 6}
Modificamos todas las funciones para extender al tipo Unit.

\section*{Ejercicio 7}
Agregamos reglas de evaluaci�n:\\


$$ \infer{(t_1,t_2) \rightarrow (v_1,t_2)}{t_1 \rightarrow v_1} $$

$$ \infer{(v_1,t_2) \rightarrow (v_1,v_2)}{t_2 \rightarrow v_2} $$


$$ fst \ (v_1,v_2) \rightarrow v_1 $$
$$ snd \ (v_1,v_2) \rightarrow v_2 $$


\section*{Ejercicio 8}
An�logamente extendimos las funciones.
\section*{Ejercicio 9}
Ver en derivaciones.pdf
\section*{Ejercicio 10}
\section*{Ejercicio 11}

\end{document}


